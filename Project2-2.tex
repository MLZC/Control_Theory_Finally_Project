{\bf Model 2-1(no terminal cost):} Our optimization problem is:
\begin{equation}
    J(u,T)=\int_{0}^{T}\ln(u+1)dt\rightarrow \max_{u}\ s.t. \eqref{cd:initial_condition}
\end{equation}
\begin{enumerate}[a)]
    \item Write down the Hamiltonian Function:
            \begin{equation}\label{eq:hamiltonian2}
                H(x,u,\psi)=\ln(u+1)+\psi (u-\delta)
            \end{equation}
    \item It's first order partial derivatives w.r.t $u$ is:
            \begin{equation}
                \frac{\partial }{\partial u}H(x,u,\psi)=\frac{1}{u+1}+\psi
            \end{equation}
             According to the first order extremality condition:
             \begin{equation}\label{eq:optc_with_psi2}
                u^*(t)=-\frac{1+\psi}{\psi}
             \end{equation}
             And $\frac{\partial^2 }{\partial u^2}H(x,u,\psi)\big|_{u=u^*}<0$, we can conclude that the Hamiltonian $H$ is concave w.r.t $u$.
    \item We substituete \eqref{eq:optc_with_psi2} into \eqref{eq:hamiltonian2} to get the maximal Hamiltonian Function:
             \begin{equation}
                \mathcal{H}(x, \psi)=H(x,u^*,\psi)=-\ln(-\psi)-(1+\psi)-\psi\delta
             \end{equation}
    \item The canonical form is writen as:
             \begin{equation}\label{eq:canonical_form2}
                 \begin{dcases}
                    \dot{x}= \frac{\partial }{\partial \psi}H(x,u,\psi)\bigg|_{u=u^*}=-\frac{1+\psi}{\psi}-\delta\\
                    \dot{\psi}=-\frac{\partial }{\partial x}H(x,u,\psi)\bigg|_{u=u^*}=0
                \end{dcases}
             \end{equation}
    \item From D.E.S \eqref{eq:canonical_form2}, it's not hard to find that $\psi(t)\equiv \psi_0$, $\psi_0$ is a constant.

    \item According to D.E.S \eqref{eq:canonical_form2} and $\psi(t)\equiv \psi_0$, we can find the optimal trajectory:
             \begin{equation}\label{eq:optx_with_psi02}
                 x^*(t)=x_0-\bigg(\frac{1+\psi_0}{\psi_0}+\delta\bigg)t
             \end{equation}
    \item From Equation \eqref{eq:optx_with_psi02} and initial condition $x(T)=x_T$, we can find:
             \begin{equation}
                \psi(t)\equiv\psi_0=\frac{T}{x_0-x_T-T(1+\delta)} \label{eq:psi2}
             \end{equation}

    \item    Substitute $\psi_0$ into Equation \eqref{eq:optc_with_psi2}, we can find the optimal control:
        \begin{equation}\label{eq:optc2}
            u^*(t)\equiv\frac{x_T-x_0}{T}+\delta,\ \text{where}\ 0\leq\frac{x_T-x_0}{T}+\delta\leq b
        \end{equation}
    \item Substitute $\psi_0$ into Equation \eqref{eq:optx_with_psi02}, we can find the optimal trajectory:
        \begin{equation}\label{eq:optx2}
            x^*(t)=x_0+\frac{x_T-x_0}{T}t
        \end{equation}
    \item During the optimal control \eqref{eq:optc2}, the total profit is:
        \begin{equation}
            V=aJ(u^*,T)=-aT\ln(-\psi_0)=aT\biggl[\ln\big(x_T-x_0+T(1+\delta)\big)-\ln T\biggr]
        \end{equation}
\end{enumerate}

{\bf Model 2-2(terminal cost):} Our optimization problem is:
\begin{equation}
    J(u,T)=\int_{0}^{T}\ln(u+1)dt-Dx(T)\rightarrow \max_{u}\ s.t. \eqref{cd:initial_condition}
\end{equation}

Same procedure as \textbf{Model 2-1} item a) to f). The difference between them is there are no boundary conditions on $\psi$ in Model 2-1. But for Model 2-2 is not.

\begin{equation}
    \psi(T)=\psi_0=-\frac{d}{dx}Dx(t)\bigg|_{t=T}=-D
\end{equation}

Hence, $\psi(t)\equiv-D$.

\begin{itemize}
    \item Substitute $\psi(t)\equiv-D$ into Equation \eqref{eq:optc_with_psi2}, we can find the optimal control:
    \begin{equation}
        u^*(t)=\frac{1-D}{D},\ \text{where}\ b\geq \frac{1-D}{D}
    \end{equation}
    \item Substitute $\psi(t)\equiv-D$ into Equation \eqref{eq:optx_with_psi02}, we get:
    \begin{equation}
        x^*(t)=x_0+\bigg(\frac{1-D}{D}-\delta\bigg)t
    \end{equation}

    Define $t^*=\frac{-x_0D}{1-D-D\delta}$.

    There are also two cases for $x$: 

    {\bf Case 1:} $u^*\geq\delta$
    \begin{equation}
        x^*(t)=x_0+\bigg(\frac{1-D}{D}-\delta\bigg)t
    \end{equation}

    {\bf Case 2:} $u^*<\delta$

    \begin{equation}
        \begin{dcases}
            x^*(t)=x_0+\bigg(\frac{1-D}{D}-\delta\bigg)t & 0\leq t\leq t^*\\
            x^*(t)=0 & t> t^*
        \end{dcases}
    \end{equation}

    \item Finally, the optimal profit is:
    \begin{itemize}
        \item For $T\leq t^*$
        \begin{gather}
            \begin{aligned}
                V=aJ(u^*,T)=-aT\ln D-a\big[Dx_0+(1-D-\delta D)T\big]
            \end{aligned}
        \end{gather}
        \item For $T> t^*$, there are no terminal costs:
        \begin{equation}
            V=-aT\ln D
        \end{equation}
    \end{itemize}

\end{itemize}